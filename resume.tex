% !TEX program = xelatex

\documentclass{resume}
%\usepackage{zh_CN-Adobefonts_external} % Simplified Chinese Support using external fonts (./fonts/zh_CN-Adobe/)
%\usepackage{zh_CN-Adobefonts_internal} % Simplified Chinese Support using system fonts

\begin{document}
\pagenumbering{gobble} % suppress displaying page number

\name{Yu Zi}

\basicInfo{
  (+1) 412-482-0674 $\bullet$
  zyu539@gmail.com $\bullet$
  183 Jeffrey Ter, Sunnyvale, CA 94089
}

\section{Education}
\datedsubsection{\textbf{Carnegie Mellon University }\textit{Master of Information System Management (BIDA)}}{Aug. 2021 -- Aug. 2022}
\begin{itemize}[parsep=0.5ex]
  \item \textbf{\textit{Highest Distinction}} of Faculty of Heinz College
\end{itemize}
\datedsubsection{\textbf{The University of Auckland }\textit{Bachelor of Software Engineering (Honours)}}{Jul. 2014 -- May. 2018}
\begin{itemize}[parsep=0.5ex]
  \item \textbf{\textit{Dean's Honours List}} of Faculty of Engineering (\textit{First Class})
\end{itemize}

\section{Skills}
\begin{itemize}[parsep=0.5ex]
  \item \textbf{Programming Languages:} Java, Python, Go, C++, C, JavaScript
  \item \textbf{Other Knowledge:} PyTorch, Spark, Kubernetes, Docker, Kafka, Samza, MySQL, HBase, MongoDB
  \item \textbf{Interests:} Software Engineering, Cloud Computing, Distributed System, Machine Learning
\end{itemize}

\section{Work Experience}
\datedsubsection{\textbf{Kakapo Technologies Limited., New Zealand } \textit{(Full-time Developer)}}{Apr. 2018 - May. 2021}
\begin{flushleft}
Develop and maintain a website for National Australia Bank to reconcile trade information across disparate systems.\linebreak It regularly extracts data from systems and identifies trades that should be the same, locating any differences
\begin{itemize}
  \item Multiple Python services that run on AWS to monitor the arrival of data from different systems and \linebreak coordinate the graph of thousands of reconciliation actions each day
  \item Parsers and Builders that handle the processing and analyzing of data in hundreds of different formats
\end{itemize}
\end{flushleft}
%\datedsubsection{\textbf{SJGTW Electrical Commercial Company, China } \textit{(Intern)}}{Jun. 2017 - Jul. 2017}
%\begin{flushleft}
%Built a tool that automatically classifies building materials given the name and model (with Keras)
%\begin{itemize}
%  \item Used Neural Network to predicate the class of material, achieving a correctness rate of over 80\%
%\end{itemize}
%\end{flushleft}
% Reference Test
%\datedsubsection{\textbf{Paper Title\cite{zaharia2012resilient}}}{May. 2015}
%An xxx optimized for xxx\cite{verma2015large}
%\begin{itemize}
%  \item main contribution
%\end{itemize}

\section{Projects}

\begin{flushleft}
\datedsubsection{\textbf{Generative Model for Survival Analysis}}{\textit{Deep Learning Project [2022]}}
Use Categorical Variational AutoEncoders + Deep Survival Machines(a fully-parametrical network) to build a model for Time-to-Event Survival Analysis (e.g., when a patient will die).

\datedsubsection{\textbf{Cloud-Based Twitter User Recommendation System}}{\textit{Cloud Computing Project [2021]}}
A cloud-based web service which recommend some close friends of a input user with similar interests to you.
\begin{itemize}
    \item Use Spark to process over 1TB Twitter dataset, clean and pre-calculate interactive \& common hashtag \linebreak \& keyword score between contact users (users who have retweeted or reply to each other)
    \item Processed Dataset stored in MariaDB on Amazon Relational Database Service
    \item A web service respond to user request was deployed on AWS EKS, reaching throughput of 16000 requests/sec,\linebreak within \$1.2/hour budget.
\end{itemize}

\datedsubsection{\textbf{Build a Naive Raft Consensus for Distributed Systems with Go}}{\textit{MIT 6.824 Project [2021]}}

\datedsubsection{\textbf{Real-Time Cabs Matching System}}{\textit{Cloud Computing Project [2021]}}
Use Kafka and Samza stream processing on EMR to match customers with cabs in real time based on distances, preferences, etc.

\datedsubsection{\textbf{Cloud-based Anomaly Route Detection System}}{\textit{Part IV Project Research [2017]}}
Worked under the supervision of \textit{Dr. Xuyun Zhang} to build a mobile application which can tell users whether \linebreak the taxi driver is taking an anomalous route
\begin{itemize}
    \item A mobile application to collect real-time trajectory data and send to back-end server for analyzing
    \item Scripts to deal with pre-processing initial dataset by using Hadoop MapReduce
    \item A cloud-based Java backend to store and normalize route data, train a prediction model periodically, \linebreak and justify the current route with the trained model, achieving a correctness rate of over 90\%
\end{itemize}
\end{flushleft}

\section{Honors}
\datedline{\textbf{\textit{\nth{11}}} in \textbf{\textit{ACM-ICPC}} Programming Contest \textbf{South Pacific Regional Final}}{2016}
\datedline{\textbf{\textit{\nth{2} Place}}, in \textbf{\textit{New Zealand Programming Contest}} (Tertiary Open Category)}{2016}
\datedline{\textbf{\textit{\nth{87}}} in \textbf{\textit{IEEEXtreme}} Programming (out of 1823 teams, Top 5\%)}{2016}
\datedline{\textbf{\textit{\nth{3} Place}}, in \textbf{\textit{New Zealand Programming Contest}} (Tertiary Intermediate Category)}{2015}

\end{document}
